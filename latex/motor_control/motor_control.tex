\documentclass [UTF8]{ctexart}
\title{电控组答题}
\author{PB25151789 刘广易}
\date{\today}
\usepackage{amsmath}
\usepackage{graphicx}
\usepackage[a4paper, left=2.5cm, right=2.5cm, top=2.5cm, bottom=2.5cm]{geometry}


\begin{document}
\maketitle{常规题目}\\
1.我想获得的收获。我希望获得的收获有以下几点,首先是学一些东西既包括知识层面的东西,比如说电控组的嵌入式方面的知识,以及一些工程领域的基础技能,比如建模,写算法。其次是在工作中想提升自己的合作能力以及解决问题的能力。\\
2.我希望自己做出的贡献。我希望我能够做到编写数据传输与可视化方案,调试设备等。 \\
3.时间规划,周一到周五每天一个小时学习相关知识,周六周日每天4-6小时调试设备编写代码。\\
4.意愿,我愿意,哪怕不能成为正式成员,我也愿意参加现场的比赛,从实战中提升自己的能力。\\
5.经历,我高中时参加过学校的机器人社团,虽然没有深度参与比赛,但是有浓厚的兴趣。我对单片机十分感兴趣,以及我有不懈钻研的毅力。\\
\\
\maketitle{题目1}\\
直流无刷电机关注的指标:\\
1)	电数据,主要是额定电压,电流,还有峰值电流。需要供电保持在一定区间内,否则电机无法运转,或对电机造成不可逆的损伤。\\
2)	机械指标,主要是扭矩和转速。扭矩体现电机的输出能力,需要功率保持在一定范围内,既不浪费也不至于损坏电机。同时扭矩决定了这个电机的输出能力,能够实现什么样的功能。转速决定了电机的旋转速度,要根据需求选择电机的转速。同时,电机的转速要与扭矩适配,扭矩越大转速越小,输出的“力道”越大,扭矩越小转速越大,低速状态下电机不稳定。\\
3)	结构参数,决定了电机能不能适配机械架构。\\
4)	参数曲线,能够直观看出在不同扭矩下,电机参数的变化\\
\includegraphics{motor.png}\\
\\
\maketitle{题目2}\\
1.	计算负载率。以一个标准帧128位来计算。负载为大疆电机每秒输出4*1000帧,输入1000帧,达妙电机每秒输入输出各五百帧,总计六千帧*128bps。除以1Mbps的波特率,也就是768000/1000000*100%=76.8%\\
2.	差分电平\\
1)	差分电平是由can收发器处理后得到的信号,由两根导线传输一个信号,单片机系统的低电平会被转换成一个1.5v一个3.5v的信号,高电平会被转换成两个3.5v的信号。这样可以通过比较两根导线之间的压强差来收发高低电平。\\
2)	如果收到干扰,一般会同时改变两根导线内的数据,一般是改变数据的电平的高低性。由两根导线的压强差来传输高低电平的差分电平会在收到干扰时保持数据的压强差,所以可以实现较长的距离传输和较高的稳定性。同时,差分电平的判断不再依赖GND电平的基准,更加的适于长距离传输信号。\\
3)	将差分电平的两股传输数据线绞在一起,减少不必要的干扰。总线两端接120欧姆的电阻,消除信号反射;总线的支线尽量要短,避免信号反射。保证两根电线的长度相等,避免信号同步方面的逻辑错误。\\
3.	在终端的两端各连接120欧姆的电阻,可以消除信号的反射,避免干扰。\\
\\
\maketitle{题目3}\\
采用一阶低通滤波(IIR)算法,将角度变化量除以检测间隔计算角速度,利用历史计算的角速度结果综合新的角速度数据来得到新的速度数据,加权度由一个常数α决定。\\
经过调参,发现在α等于0.03左右时,既能保证极值点的稳定,也能保持曲线的一定平滑度。\\
公式如下\\
$$\Delta\theta[t]=\theta[t]-\theta[t-1] $$
$$if\ \Delta\theta[t]<-2\pi,\Delta\theta[t]=\Delta\theta[t]+2\pi $$
$$if\ \Delta\theta[t]>2\pi,\Delta\theta[t]=\Delta\theta[t]-2\pi$$
$$\omega[t]=\frac{\Delta\theta[t]}{\Delta t}$$
$$\bar{\omega}[t]=\alpha\omega[t]+(1-\alpha)\bar{\omega}[t-1]$$

\begin{figure}
\centering \includegraphics[width=0.5\textwidth]{motor_speed0.01.jpg}\caption{\alpha=0.01}
\centering\includegraphics[width=0.5\textwidth]{motor_speed0.03.jpg}\caption{\alpha=0.03} \\
\centering\includegraphics[width=0.5\textwidth]{motor_speed0.05.jpg}\caption{\alpha=0.05} 
\centering\includegraphics[width=0.5\textwidth]{motor_speed0.1.jpg}\caption{\alpha=0.1}\\
\centering\includegraphics[width=0.5\textwidth]{motor_speed0.2.jpg}\caption{\alpha=0.2}
\end{figure}

\maketitle{题目4}
优先级设置为底盘数据大于超级电容大于 云台角度。
以超级电容更新时间为基准,每二十毫秒检测是否有底盘模式数据,如果没有则发送云台角度数据,如果有则发送底盘是否处于小陀螺模式数据。当发送四次后,第一百毫秒检测是否有底盘数据,如果有,发送底盘数据,20ms后发送电容数据,之后循环往复。\\



\end{document}