\documentclass[12pt,a4paper,UTF8]{ctexart}
\usepackage{amsmath,amssymb}
\usepackage{amsmath,amssymb} 
\usepackage{geometry}  
\usepackage{titlesec}  


\geometry{top=2.5cm,bottom=2.5cm,left=2.5cm,right=2.5cm}

\title{从戴德金分割到实数的完备性:实数理论的逻辑化构造}
\author{PB25151789 刘广易}
\date{\today}

\titleformat{\section}{\bfseries\Large}{\thesection}{1em}{}
\titleformat{\subsection}{\bfseries\large}{\thesubsection}{1em}{}

\begin{document}
\maketitle

\begin{abstract}
实数是数学分析学的基石。但在数学史上,最令人惊奇的事实之一,便是实数系的逻辑基础竟然迟至十九世纪后半叶菜建立起来。
在那时以前,即使正负有理数与无理数的最简单性质也还没有建立。直到1872年,分别由德国数学家戴德金、康托尔、维尔斯特拉斯建立的三大实数理论才弥补了这个微积分发展过程中遗留下来的巨大缺陷。
其通过严密逻辑性推导建立的实数理论,完善了分析学的底层架构,更揭示了实数系的完备性本质。
本文比较戴德金派的“分割理论”与教材中提到的Heine“柯西序列”两种构造方式的逻辑结构与哲学意涵,探讨完备性在数学与思想中的核心地位。
\end{abstract}


\section*{一、历史的回顾:连续的幻想与有理数的危机}
    \indent实数系的逻辑结构问题在19世纪后半叶被数学家们重点关注,解决无理数的定义问题是当之无愧的难点。
    但当我们回顾19世纪数学大叔们所建立的五i书理论,除了一个新的观点外,其余的 所有内容早在Euclid《原本》第五卷中已经揭示。Euclid给有理数$m/n$分为三类,
    分别是小于$a/b$、等于$a/b$、大于$a/b$。但是他的逻辑是有缺陷的,他并没有定义一个不可公度的比。而千年后,Dedekind所构建的实数理论,恰恰基于这个缺陷。
    正因为这个定义上的疏忽,当Euclid沿用几何上的直观处理边长为1的正方形的对角线长度$\sqrt{2}$时,他深深陷入了理性与直觉的对立。
    这不仅仅成为了第一次数学危机的滥觞,更为千年后第二次数学危机的酝酿埋下了伏笔。\\
    \indent在十七世纪,牛顿和莱布尼茨创立了让数学界翻天覆地的微积分,以无穷小为显微镜,洞察了曲线下的面积,长度,乃至刻画了行星的轨迹,万物的运动。
    但这无疑为数学的地基筑下了倾斜的地基:什么是无穷小,什么是极限?这些在十八世纪哲学家哲学家George Berkeley笔下被称作“死去的量的幽灵”。揭示出当时分析学的逻辑困境。\\
    \indent十九世纪中叶,极限与无穷的理论缺陷所引发的第二次数学危机的到来数学家们的重视,在此背景下,
    戴德金和维尔斯特拉斯等人以不同路径重建了实数的逻辑,以“序列收敛”定义极限,完成了“为连续赋予逻辑形式“的使命。回答了“什么是数”的问题,将数学研究带上了严谨与标准的康庄大道。
    
\section*{二、现世的倒影:有理数的稠密与缺陷}
    \indent有理数集$\mathbb{Q}$已经足够稠密,
    容易知道,$\forall a,b \in\mathbb{Q} ,a<b \Rightarrow\exists c\in\mathbb{Q},a<c<b$。
    这种性质在数学上称作稠密性,即无论两个有理数之间间隔多么的小,都能在中间找到另一个有理数。
    自然的,我们会觉得,有理数一定能够填满整个数轴,使之没有缝隙。但是十分诡异的是,有理数并不具备完备性(Completeness),也就是说,有理数并不连续。
    比如说,考虑集合$A=\{a\in \mathbb{Q}|q^2<2\}$,它在有理数中显然有上界,但并不存在一个有理数作为它的上界。说明$\sqrt{2}$这个数字在有理数域内缺席的,这便是看似完整的体系中被分开的裂隙。哪怕A集合中的数字多么接近$\sqrt{2}$,但都无法等于它,这意味着有理数是不连续的,是有着严重缺陷的,是不完备的。
    \indent什么是完备性?完备性,这是一个后世对于度量空间给出的概念,乍看上去让人摸不到头脑。
    从字面上看,完备性便是完成,完整,给人一种稳妥的感觉。
    从定义上看,任何Cauchy数列都收敛的度量空间被称为完备的,
    但这个定义对于十九世纪的数学家来说还有些超前。让我们看向最开始的数轴,能否精确表达出实数轴上的一个点?这个问题看似很容易,但回答起来并不简单。
    % 想表示哪个点就在数轴上标出对应的数字就好了。但是事情远远没有这么简单,容易知道,无理数有无穷多个,更进一步的,不能用代数方程的根来表示的无理数同样有无穷多个,更不能每一个都找个符号来表示。
    这个问题在当时,哪怕是高斯、拉格朗日这种顶级数学家都对此望而却步。对这个问题的求索,便是翻开新一页数学篇章的钥匙,而正是Dedekind找到了它,带领数学迈向新的阶段。
    
\section*{三、戴德金分割:有理数域的断裂创造实数域的连续}
    \indent 1872年,Dedekind用他的分割理论解决了对于实数的定义问题。他采用对有理数域$\mathbb{Q}$的分割来定义实数。\\
    \indent 定义 把有理数域$\mathbb{Q}$分割为两个非空且不相交的子集
    $A$和$A'$,$\mathbb{Q}= A\cup A',A\cap A'=\phi$,使得对任意的
    $a\in A$和$a'\in A'$,有$a<a'$。集合$A$成为分割的下组,集合$A'$称为集合的上组,并将这种分割记为$ A\mid A'$。\\
    \indent在上述定义下,A和$A'$ 在$\mathbb{Q}$中互为余集:$A=\mathbb{Q}\setminus A'$,$A'=\mathbb{Q}\setminus A$。即对任意的集合$A\subset\mathbb{Q}$,$A$在$\mathbb{Q}$中的余集为\\
    $$\mathbb{Q}\setminus A=\{a\mid a\in\mathbb{Q},\,\text{但}\,a\notin A\}.$$
    \indent从数轴上看,一个分割实际上是把数轴上所有有理点分成左边一部分(即是下组)和右边一部分(即是上组)。因此就得出分割的三种可能:\\
    \indent(1) 在下组$A$内无最大数,而在上组$A'$内有最小数$r$,
    $$A = \{a\mid a\in \mathbb{Q},\,a<r\},\quad A'=\{a'\mid a'\in\mathbb{Q},\,a'\geq r\};$$
    \indent(2) 在下组$A$内有最大数,而在上组 $A'$内无最小数 $r $,
    $$A = \{a\mid a\in\mathbb{Q},\,a\leq r\},\quad A'=\{a'\mid a'\in\mathbb{Q},\,a'>r\};$$
    \indent(3) 在下组 $A$内无最大数,而在上组 $A'$内也无最小数,例如
    $$A = \{a\mid a\in\mathbb{Q},\, a<0;\,\text{或}\,a\geq0\,\text{但}\,a^2<2\},$$
    $$A' = \{a'\mid a'\in\mathbb{Q},\,a'>0\,\text{且}\,a'^2>2\}.$$
    \indent 可以用反证法证明,不存在第四种分割。\\
    \indent 当分割形如第三种时,A中没有最大的有理数,$A'$里也没有最小的有理数,正如最开始的例子$A:x<\sqrt{2},A'>\sqrt{2}$。虽然两个集合$A$和$A'$的并集包含了所有的有理数,
    但这个并集并不包含所有实数,$\sqrt{2}$这个数字正好被夹到了两个区域中间,所以我们可以从两个区域的任意一个出发,用有理数来逼近它。同理可知每一个形如(3)的分割都对应着一个新的数。
    虽然有理数的稠密性并不代表有理数集的连续,但恰恰是这个性质使得实数集能够被构造。正是这个性质决定了,每一个无理数周围无论多么小的邻域,都能够找到一个有理数,所以可以使用逼近的方式构造无理数。\\
    综上所述,戴德金分割导出了实数的连续与完备,弥补了微积分基础的不稳,使整个微积分有了一个坚固的连续统基础。
\section*{四、从柯西序列看戴德金分割的等价性}
\indent $\pi$和e作为最被人熟知的无理数,自然是实数领域研究的最好素材。
    在由莱布尼兹和惠更斯发现$\pi=a\cdot\sum_{n=1}^{\infty}(-1)^{n-1}\frac{1}{2n-1}$,Euler发现$e=\sum_{n=0}^{\infty}\frac{1}{n!}$后,人们惊奇的意识到
    无理数竟然可以由有理数的累加取得。于是,自然而然地想到,能不能用无穷级数来构造所有的实数呢。众所周知,无穷级数既可以发散也可以收敛,我们只需要那些收敛的级数。
    这里便需要教材中提到的柯西审敛准则
    $\forall m,n >N(\epsilon),\left|x_{m}-x_{n}\right|<\epsilon$。
    恰恰是这样一个简单的判别式,让实数集具有了完备性。
    虽然柯西早在1821年便提出了柯西序列的概念,
    但直到1872年才由Heine、Cantor完全形式化阐明。\\
    对于上一节提到的$\sqrt{2}$,可以由迭代逼近的方法构造:
    $$a_{1}=1,a_{n+1}=\frac{a_{n}+\frac{2}{a_{n}}}{2}$$
    可以证明该序列逐渐逼近$\sqrt{2}$。这样,通过极限逼近,实数$\sqrt{2}$在逻辑上被“捕获”。
    这样,将实数由静态的点转化为动态收敛的数列的终点。\\
    \indent Dedekind分割告诉我们“存在”;柯西序列告诉我们“如何到达”通过两种方法的比较,我们理解了完备性不仅仅是数学性质,同时也是分析学的核心逻辑基础,为后续的极限、连续函数、微积分定理提供了坚实支撑。\\
    \indent 但同样的,戴德金分割亦是通过动态的逼近完成了从有理数到实数的过度;而柯西收敛准则则是通过动态的收敛得到了静态的极限值。在数学上可以证明,这两种极限定义形式是等价的。\\
\section*{五、终焉:实数理论建立的意义}
    \indent 从戴德金分割到柯西序列,我们见证了数学家如何在离散与连续之间架起理性的桥梁。
    实数体系不仅是分析学的基础,更是思想对连续性问题的逻辑回应。\\
    \indent 戴德金分割强调逻辑存在性:实数是集合的划分,每一个缺口都被形式化填补;柯西序列强调逼近与过程:实数是极限的机制,动态体现连续;完备性将二者统一,使得极限、连续、导数和积分在数轴上严格成立。换言之,连续不是偶然的物理现象,而是在数学上由理性逻辑推理精心构造的产物。
    从有理数集合造成的不连续到实数域的完备;从可数的有理数到不可数的实数,实数将离散的有理点连缀成连续的数轴,一代代数学家为之付出的心血更体现了理性思辨的魅力。
\begin{thebibliography}{99}
\bibitem{dedekind1872} Dedekind, R. (1872). *Stetigkeit und irrationale Zahlen*. Vieweg.
\bibitem{cauchy1821} Cauchy, A.-L. (1821). *Cours d’analyse*. Paris: Debure.
\bibitem{weierstrass1861} Weierstrass, K. (1861). *Über die Grundlagen der Analysis*. Lecture Notes, Berlin.
\bibitem{bili_cauchy_series}
“柯西数列,成全了微积分的一个定义,以及实数域的自我救赎”Bilibili视频https://www.bilibili.com/video/BV1Ax4y1C7wD/
\bibitem{bili_dedekind_cut_nested_intervals}
“戴德金分割怎么割?康托尔区间套怎么套?”,Bilibili视频https://www.bilibili.com/video/BV16f421q7tq/
\bibitem{RenXinx}
任辛喜. 《贝克莱及其对早期微积分的批评》[J].
山西师范大学数学与计算机科学系, 山西临汾041004.
\bibitem{ChengYi}
数学基础选讲[M] 程艺 编著.
\end{thebibliography}

\end{document}
