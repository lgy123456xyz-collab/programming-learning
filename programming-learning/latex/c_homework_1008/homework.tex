\documentclass{article}
\usepackage{geometry}  
\usepackage[UTF8]{ctex}
\usepackage{listings}
\usepackage{xcolor}
\usepackage{tikz}
\usetikzlibrary{shapes.geometric, arrows}

\lstset{
    frame=single,               % 框住代码
    backgroundcolor=\color{gray!10}, % 背景色
    basicstyle=\ttfamily,       % 字体
    keywordstyle=\color{blue},  % 关键字颜色
    commentstyle=\color{green}, % 注释颜色
    numbers=left,               % 行号
    numberstyle=\tiny,          % 行号字体
    breaklines=true             % 自动换行
}
\geometry{top=2.5cm,bottom=2.5cm,left=2.5cm,right=2.5cm}


\title{计算机作业 25/10/08}
\author{PB25151789 刘广易}
\date{\today} 

\begin{document}
\maketitle
\section{P34 T4}
\indent 代码:

\begin{lstlisting}
#include<stdio.h>
int main()
{
while(1)
{
    double m,kg,bmi;
    printf("Please input your height(m) and weight(kg),input \"0 0\" to quit.");
    scanf("%lf %lf",&m,&kg);
    if(m==0&&kg==0)
    break;
    bmi=kg/m/m;
    bmi>25?(printf("You are so big\n")):(bmi<20?printf("You are so small\n"):printf("You are normal\n"));
}
    return 0;
}
\end{lstlisting}


\indent 伪代码 \\

重复执行 \\
    从键盘输入身高(m)、体重(kg)\\
    if m=0 and kg=0,退出\\
    算BMI=体重/(身高*身高)\\
    $BMI>25?$输出超重:($BMI<20?$输出瘦弱:输出正常)\\




\indent 流程图:\\
\begin{tikzpicture}[node distance=2cm, >=stealth]

\tikzstyle{startstop} = [rectangle, rounded corners, minimum width=3cm, minimum height=1cm, text centered, draw=black, fill=red!20]
\tikzstyle{process} = [rectangle, minimum width=3cm, minimum height=1cm, text centered, draw=black, fill=orange!20]
\tikzstyle{decision} = [diamond, aspect=2, text centered, draw=black, fill=green!20]
\tikzstyle{arrow} = [thick,->,>=stealth]
\tikzstyle{io} = [trapezium, trapezium left angle=70, trapezium right angle=110, minimum width=2cm, minimum height=1cm, text centered, draw=black,fill=blue!20]

% 节点定义
\node (start) [startstop] {开始};
\node (input) [io, below of=start] {输入身高、体重};
\node (process0) [process, below of=input, yshift=-0.5cm] {计算BMI};
\node (decide) [decision, below of=process0, yshift=-0.5cm] {$BMI>25$?};
\node (decide1) [decision, below of=decide, yshift=-0.5cm] {$BMI<20$?};
\node (process2) [io, right of=decide, xshift=3.5cm] {输出超重};
\node (process4) [io, below of=decide1, yshift=-0.5cm] {输出正常};
\node (stop) [io, below of=process4] {输入身高、体重};
\node (process3) [io, left of=decide1, xshift=-3.5cm] {输出瘦弱};

% 连接箭头
\draw [arrow] (start) -- (input);
\draw [arrow] (input) -- (process0);
\draw [arrow] (process0) -- (decide);
\draw [arrow] (decide) -- node[anchor=east] {NO} (decide1);
\draw [arrow] (decide) -- node[anchor=south] {YES} (process2);
\draw [arrow] (decide1) -- (process4);
\draw [arrow] (process4) -- (stop);;
\draw [arrow] (process2) |- (stop);
\draw [arrow] (decide1) -- (process3);
\draw [arrow] (process3) |- (stop);
\draw [arrow] (stop.south) -- ++(0,-1)   
                 -- ++(-9,0)             
                 -- ++(0,11)              
                 -- ++(3,0)              
                 -- (process0.west);    

\end{tikzpicture}

\section{P34 T7}
\indent 流程图:\\
\begin{tikzpicture}[node distance=2cm, >=stealth]

\tikzstyle{startstop} = [rectangle, rounded corners, minimum width=3cm, minimum height=1cm, text centered, draw=black, fill=red!20]
\tikzstyle{process} = [rectangle, minimum width=3cm, minimum height=1cm, text centered, draw=black, fill=orange!20]
\tikzstyle{decision} = [diamond, aspect=2, text centered, draw=black, fill=green!20]
\tikzstyle{arrow} = [thick,->,>=stealth]
\tikzstyle{io} = [trapezium, trapezium left angle=70, trapezium right angle=110, minimum width=2cm, minimum height=1cm, text centered, draw=black,fill=blue!20]



\node (start) [startstop] {开始};
\node (input) [io, below of=start] {输入数据};
\node (process0) [process, below of=input, yshift=-0.5cm] {加到sum上;计输入数};
\node (decide) [decision, below of=process0, yshift=-0.5cm] {输入次数=4?};
\node (process2) [process, below of=decide, yshift=-0.5cm] {计算平均值};
\node (output) [io, below of=process2, yshift=-0.5cm] {输出平均值};
\node (stop) [startstop, below of=output] {结束};


\draw [arrow] (start) -- (input);
\draw [arrow] (input) -- (process0);
\draw [arrow] (process0) -- (decide);
\draw [arrow] (decide) -- node[anchor=east] {YES} (process2);
\draw [arrow] (decide.west)node[anchor=south] {NO} 
                            -- ++(-3,0)
                            -- ++(0,5)
                            -- ++(3,0)
                            -- (input.west);
\draw [arrow] (process2) -- (output);
\draw [arrow] (output) -- (stop);;



\end{tikzpicture}

\section{P91 T5}
\indent long long int占据64位八字节的数据 可以表示$-2^{63}$到$2^{63}-1$\\
\indent 当整数运算结果超出long long int时可以将整数以十进制储存在数组中,写算法对数组内的数据进行四则运算\\
\indent  \\
\section{P91 T7}
\indent 
\begin{lstlisting}
#include<stdio.h>
#define A m==0&&kg==0) break;
#define B bmi=kg/m/m;
#define C printf("You are so big\n")):(bmi<20?printf("You are so small\n"):printf("You are normal\n"));
#define D printf("Please input your height(m) and weight(kg),input \"0 0\" to quit.");
#define E scanf("%lf %lf",&m,&kg);
int main()
{
while(1)
{
    double m,kg,bmi;
    D
    E
    if(A
    B
    bmi>25?(C
}
    return 0;
}
\end{lstlisting}

\indent 编译不能通过的写法:\\

\begin{lstlisting}
    #define { (
    #define } )
\end{lstlisting}
\indent因为包含C的关键字\\
\end{document}