\documentclass[12pt,a4paper,UTF8]{ctexart}
\usepackage{amsmath,amssymb}
\usepackage{amsmath,amssymb} 
\usepackage{float}
\usepackage{geometry}  
\usepackage{titlesec}  
\usepackage{graphicx}

% \usepackage{zhnumber}
% \renewcommand{\thesection}{\zhnum{section}}
\geometry{top=2.0cm,bottom=2.0cm,left=2.5cm,right=2.5cm}

\title{桥牌课程计算及登记系统 项目报告}
\author{PB25151789 刘广易 PB25151800 许昊天}
\date{\today}

\titleformat{\section}{\bfseries\Large}{\thesection}{1em}{}
\titleformat{\subsection}{\bfseries\large}{\thesubsection}{1em}{}

\begin{document}
\maketitle

\begin{abstract}
本项目设计并实现了一款为中国科大桥牌课程的教师和学生提供便利的计算及登记系统。
该系统旨在简化桥牌课程中的成绩计算和学生信息管理流程,提高工作效率。
针对性的解决了桥牌学习与练习中算分复杂、成绩统计滞后以及缺乏数字化管理等痛点,提供了一套完整的解决方案。
系统核心功能包括了基于国际桥牌标准的IMP算分引擎、集成Elo等级分系统的动态排行榜、基于Ttie树的模拟叫牌辅助系统、以及自动化的成绩申报与审核流程。
通过使用现代编程技术和数据库管理系统,我们确保了系统的稳定性和易用性。系统采用Python Flask后端配合Javascript前端架构,实现了管理员与普通用户的双视角管理。
交互设计上采取了扁平化的直观界面,确保用户能够轻松上手。系统还支持多终端访问,方便教师与学生随时随地使用。
项目的实施不仅提升了课程管理的效率,也为未来类似课程的管理提供了参考和借鉴。
同时通过模块化设计为未来的“ocr牌例识别”、“智能叫牌建议”等功能留出了扩展接口,是一项具有较高鲁棒性与推广意义的校园实用工具开发实践。

\end{abstract}
\section{项目概述}
    \indent桥牌作为一项高度依赖逻辑推理与精密计算的智力运动,在高校中拥有广泛的受众。然而,线下练习与学习时常常面临IMP分计算繁琐、叫牌规则记忆难度大、学员之间PK成绩难以量化管理等问题。
    本项目开发的“桥牌综合辅助工具集”,旨在为中国科大桥牌课程的教师与学生提供一站式的计算及登记解决方案。
    本项目旨在为中国科学技术大学PE1502.01《桥牌基础讲座与技巧》课程量身定制一套集教学分株、成绩自动化管理及竞技练习自动化工具与一体的综合性实用工具系统。
    在实际教学场景中,该课程的期末成绩评价体系高度依赖于学生参与的多次PK赛与挑战结果,目前采用的人工记录方式虽然在运行,但在处理复杂数据时仍存在一定的优化空间,
    给授课老师带来了巨大的统计压力,也导致学生难以实时获取个人成绩反馈,若能提升统计效率,将为师生提供更顺畅的教学体验。针对这些核心痛点,本系统通过搭建web端应用,实现了复杂计分规则的数字化封装,提供了一套学生自主申报成绩到管理员一键审核的
    自动化流水线,极大提升了教学管理效率并确保了数据的准确性。引入了叫牌提示功能,通过对叫牌上下文的实时感知,为初学者提供合规的逻辑建议,有效降低了新手的上手门槛。
    同时,内置的动态排名系统能够根据经济结果自动调整学生积分,激发了课堂的竞技活力。此外,还接入了双明手分析求解模块。该模块能够针对特定拍剧进行穷举分析,计算出在完美信息下的最优打法与
    最大可能赢墩数,为学生在赛后复盘和牌技研讨交流中提供了科学的决策参考。
    
    \indent在用户体验设计上,工具遵循简洁直观的原则,采用响应式布局优化,确保了导航菜单栏在多端设备上的整齐排列。并结合现代化的交互设计理念,提升了用户的操作便捷性与视觉舒适度。针对一系列操作流程,进行了针对性的优化,显著提升了操作的鲁棒性与确定感。
    本作品不仅完整覆盖了课程的核心需求,还作为一款具有高稳定性的Demo演示版本,为未来桥牌教学的数字化转型提供了极具参考价值的实践案例。

    \indent为了确保系统在郑老师课程高频使用场景下的流畅性与鲁棒性,本项目对关键性能指标进行了深度优化分析。在算法检索方面,通过对叫牌序列建立Trie树结构,将查询复杂度降低到O(k),使得系统能够在毫秒级响应用户请求。接入的高效双明手求解引擎在1秒钟内完成计算。
    数据库设计上,采用了SQLite数据库,借助其轻量化且易于部署的优点,实现了对用户信息、比赛记录及积分数据的高效管理。通过合理的索引设计与查询优化,确保了在高并发访问下的数据一致性与快速响应。
    前端性能方面,针对关键交互逻辑采取了双击验证等方式,提升了用户操作的确定感与系统的鲁棒性。


\section{系统设计}
\subsection{总体架构}
\indent为了确保系统具备良好的跨平台兼容性并最大限度降低师生的部署与使用成本,本系统采用了经典的B/S\(Browser/Server\)架构设计。
前端使用HTML、CSS和JavaScript构建响应式界面,负责交互与响应式渲染,确保在不同设备上均有良好体验。
后端采用Python的Flask框架,负责处理业务逻辑和与数据库的交互。数据存储方面,选择
SQLite数据库以简化部署和维护工作。通过RESTful风格的API接口高效处理复杂的桥牌逻辑计算与数据持久管理需求。
通过将系统核心逻辑部署于个人计算机服务器端,并通过成熟的内网穿透技术实现稳定可靠的公网访问能力,确保用户无需安装特定客户端即可通过手机或电脑浏览器随时访问。
基于此,这种架构不仅提升了系统的维护效率,也为未来潜在的功能扩展和模块化升级奠定了坚实的底层基础。
\subsection{核心模块划分}
\indent系统功能被精细化的划分为四个互相协作的核心模块,以满足课程PE1502.01的多元化需求。首先是\textbf{账户权限与管理模块},通过后端的Session校验机制,其能够准确识别用户身份并实施切换管理界面与普通用户界面,确保数据访问的安全性。
其次是\textbf{成绩管理模块},该模块不仅支持学生端的异步成绩申报,还为老师提供了全面的审核工作流,涵盖了记录的审核、手动修改、删除功能,并通过自动化的积分计算与排行榜显示,大幅提高了学生参与竞技的积极性。
第三是\textbf{计算引擎模块},作为系统的逻辑核心,它集成了从单副牌定约得分计算到IMP转换,再到基于国际象棋Elo算法的等级分演变逻辑。
最后是\textbf{智能辅助模块},该模块通过接入双明手分析接口、实现Trie树结构的叫牌提示功能,为学生提供了科学的决策支持与学习辅助,显著降低了学习门槛并提升了实战能力,将原本深奥的桥牌决策过程具象化、数字化,为教学提供了强有力的辅助支持。
\subsection{关键算法}
\indent在系统的核心计算引擎模块中,集成了多项关键算法以满足桥牌课程的复杂需求。首先是\textbf{IMP分数计算算法},该算法基于国际桥牌标准,将每局比赛的得分转换为国际匹配点(IMP),通过预定义的分差与IMP对应表,实现了高效且准确的分数转换。
其次是\textbf{Elo等级分算法},该算法动态调整学生的等级分数,并根据实际结果进行分数更新,激励学生积极参与竞技。其中,\textbf{动态积分重算算法}是本系统的一大亮点,能够在管理员对成绩进行修改或删除后,自动触发相关比赛记录的重新计算,系统会重溯该时间点之后的所有对局记录并重新计算Elo等级分,确保排行版数据的准确性与客观性。
此外,\textbf{Trie树叫牌提示算法}通过构建高效的前缀树结构,叫牌提示信息的快速查询与匹配,能够在毫秒级响应用户的叫牌请求,为初学者提供合规且合理的叫牌建议。此算法通过将十六万条约定叫牌序列预先加载至内存,并构建Trie树结构,实现了O(k)的查询复杂度,极大提升了系统的响应速度与用户体验。相比字典序列遍历查找或二分查找,Trie树结构在处理大量前缀匹配查询时表现出更优的性能,将上下文感知的时间大幅压缩,实现了时间与空间的双赢。
同时,系统集成的双明手求解工具,通过蒙特卡罗模拟与穷举分析,借助牌局信息,计算出在完美信息下的最优打法与最大可能赢墩数,为研究复盘提供科学数据参考。
\subsection{界面设计}
\indent系统界面设计遵循简洁直观的原则,采用了响应式布局,使界面能够在手机与电脑端都获得良好的显示效果。导航菜单栏被设计为固定位置,确保用户在不同页面切换时仍能够方便访问各个功能模块。
对菜单栏的设计,采用了图标与文字结合的方式,提升了视觉灵敏度,同时遵循了“功能逻辑分层”的原则:第一排集中展示排行榜算分器等公共工具,第二排则根据登录身份展示申报或审核等操作性功能。
为了提升交互的鲁棒性,系统在管理员端引入了双击确认交互逻辑,有效防止了审核过程中的误触。并为提交成绩设计了实时反馈,确保用户对操作结果有清晰的认知。
\subsection{开发语言与开发工具}
\indent本系统的后端采用Python语言,利用Flask框架构建轻量级Web服务器,处理业务逻辑与数据库交互。前端则使用HTML、CSS和JavaScript,实现响应式设计与动态交互效果。
数据库方面,选择了SQLite作为数据存储解决方案,因其轻量化且易于部署,适合本项目的需求。开发过程中,使用了Visual Studio Code作为主要的代码编辑器,结合Git进行版本控制与协作开发。

\indent 在技术选型方面,本项目基于实用性与开发效率的平衡,选用了Python作为后端核心开发语言,并依托其强大的生态系统和Flask轻量级框架构建了稳定的业务中枢。数据库层面,考虑到教学场景下的数据规模和可移植需求,
系统采用了SQLite关系型数据库,因其无需安装独立的数据库服务器,同时只需存储于一个文件,方便打包和移动的优点,极大简化了数据库部署与初始数据导入工作的复杂度。
前端技术栈则选用了HTML5、CSS和JavaScript,这不仅保证了在低功耗设备上的流畅运行,也通过高度自定义的样式代码实现了两排式导航栏等特定布局的需求。
此外,项目开发中利用生成式AI工具辅助编写代码片段,在复杂换算算法的逻辑校验与跨语言翻译方面提供了显著帮助,提升了开发效率与代码质量。

\section{系统测试及运行结果}
    \subsection{测试方案与鲁棒性验证}
    \indent为了验证系统的稳定性与鲁棒性,本项目实施了多维度的鲁棒性测试方案。针对核心的\textbf{叫牌提示逻辑与IMP计算模块},我们利用生成式AI工具自动化生成了多种极端定约组合的测试用例库,并将其与人工核算数据进行交叉比对,结果显示逻辑准确率达到100\%;
    测试样例见test\_cases.txt文件。此外,针对\textbf{用户交互流程},我们设计了多种异常操作场景,如重复提交、网络中断等,确保系统在面对非预期输入时仍能保持稳定运行。
    在弱网络条件下,系统通过异步处理流程确保了ui界面的流畅响应,避免了因后端计算延迟导致的页面假死现象,确保了课堂教学环境下的使用体验。
    \subsection{测试结果}
        \begin{figure}[H]
        \centering 
        \includegraphics[width=1.3\textwidth]{test_function.png}
        \caption{test\_function}
        \end{figure}

    \subsection{工具运行界面截图}
        \begin{figure}[H]
            \centering \includegraphics[width=1\textwidth]{display.png}\caption{排行榜展示}
            \centering \includegraphics[width=1\textwidth]{display2.png}\caption{算分工具}
        \end{figure}
        \begin{figure}[H]
            \centering \includegraphics[width=1\textwidth]{display3.png}\caption{接入的双明手求解器}
        \end{figure}


        \begin{figure}[H]
    \centering
    \begin{minipage}{0.48\textwidth}
        \centering
        \includegraphics[width=\textwidth]{display4.png}
        \caption{叫牌提示工具}
        \label{fig:left}
    \end{minipage}
    \hfill 
    \begin{minipage}{0.48\textwidth}
        \centering
        \includegraphics[width=\textwidth]{display5.png}
        \caption{不同的上下文显示不同的结果}
        \label{fig:right}
    \end{minipage}
\end{figure}
        \begin{figure}[H]
            \centering \includegraphics[width=1\textwidth]{display6.png}\caption{管理员统计}
        \end{figure}

        \begin{figure}[H]
            \centering \includegraphics[width=1\textwidth]{display7.png}\caption{同学提交结果}
        \end{figure}

        \begin{figure}[H]
            \centering \includegraphics[width=1\textwidth]{display9.png}\caption{管理员审核}
        \end{figure}
        
        \begin{figure}[H]
            \centering \includegraphics[width=1\textwidth]{display8.png}\caption{同学查看结果}
        \end{figure}
\section{总结与反思}
    \subsection{项目创新点}
    \indent本项目在桥牌教学辅助工具的设计与实现上展现了多项创新点。首先,桥牌运动的自身小众属性,使得现有市场上缺乏针对高校桥牌课程的综合性数字化管理工具,针对桥牌学习的实用性辅助工具也较为匮乏。
    本系统通过集成IMP计算、Elo等级分管理、叫牌提示与双明手分析等多功能于一体,填补了这一市场空白,提供了较为完善的解决方案。
    
    \indent同时,在技术实现上,针对此系统深度贴合教学实际的定制化设计,相较于通用性工具有较为显著的创新型:
    \begin{itemize}
        \item \textbf{动态积分重算机制}:当管理员对历史成绩进行修改或删除时,系统能够自动触发相关比赛记录的重新计算,确保排行榜数据的实时准确性。这一机制有效解决了传统系统中数据一致性难以保障的问题。
        \item \textbf{Trie树叫牌提示算法}:通过构建高效的前缀树结构,实现了对十六万条约定叫牌序列的快速查询,显著提升了用户体验。该算法在处理大量前缀匹配查询时表现出优越的性能,极大缩短了响应时间。
        \item \textbf{双明手求解集成}:将复杂的双明手分析工具无缝集成至系统中,为学生提供科学的决策支持与学习辅助,降低了学习门槛并提升了实战能力。
    \end{itemize}
    这些创新点不仅提升了系统的实用性与用户体验,也为未来桥牌教学的数字化转型提供了有价值的参考与借鉴。
    
    \subsection{难点与解决方案}
    \subsubsection{前后端状态同步与反馈}
    \indent在系统开发过程中,前后端状态同步与反馈机制的实现成为一大技术难点。由于桥牌计算涉及复杂的业务逻辑,确保用户操作能够及时反映在界面上,且数据一致性得到保障,是提升用户体验的关键。
    为解决这一问题,我们引入了CSS关键帧动画,通过文字闪烁或重复出现等视觉反馈,向用户明确传达操作结果,增强了交互的确定感。
    \subsubsection{积分规则的鲁棒性}
    \indent桥牌积分计算规则复杂多变,如何确保系统在面对各种边界情况时仍能准确计算积分,是本项目的一大挑战。为此,我们深入分析了Elo等级分的计算逻辑,设计了详尽的测试用例,涵盖了各种极端情况。
    通过自动化测试与人工核验相结合的方式,确保了积分计算模块的鲁棒性与准确性。
    \subsubsection{Trie树的高效实现}
    \indent实现Trie树结构以支持高效的叫牌提示查询,是本项目的技术难点之一。面对\textbf{十六万条约定叫牌序列},如何在保证查询速度的同时,控制内存占用,是我们需要解决的问题。在最初的实现中,采用了简单的字典结构,导致查询效率低下。
    不仅需要占用1GB以上的内存空间,且查询时间较长,影响用户体验。为此,我们设计了基于二分查找的搜索算法,显著提高了查询效率,同时将内存占用控制在合理范围内。之后,了解到树形搜索算法的优势,最终采用Trie树结构,实现了O(k)的查询复杂度,大幅提升了系统的响应速度与用户体验。
    \subsection{心得体会}
    \indent通过本次项目的开发与实施,我们深刻体会到了团队协作与技术创新在实际应用中的重要性。在项目过程中,面对复杂的业务需求与技术挑战,我们不断探索与尝试,最终实现了一个功能完善且用户友好的桥牌教学辅助工具。针对特定的需求与处理逻辑,我们学习了新的开发工具,编程语言与算法设计方法,提升了自身的技术能力。

    \indent此外,我们也认识到理论知识与实际应用之间的差距。通过将课堂上学到的编程与算法知识应用于实际项目,我们不仅加深了对这些知识的理解,也提升了解决实际问题的能力。这种实践经验对于我们未来的学习与职业发展具有重要意义。
    面对实际的需求,我们不仅要具备扎实的技术基础,还需要具备良好的沟通与协作能力,以确保项目的顺利推进与成功实施。

    \indent进一步的,我们了解到优秀的实用工具不仅需要底层算法的支撑,更需要对实际需求的把握。若是没有对桥牌课程教学场景的深入理解,我们无法设计出如此贴合用户需求的功能模块。
    
    \subsection{项目不足与改进方向}
    \indent尽管本项目在桥牌教学辅助工具的设计与实现上取得了一定的成果,但仍存在一些不足之处,未来有待改进与提升。首先,受限于开发时间,OCR牌例识别功能尚未实现,未来计划引入图像处理与机器学习技术,实现对实物牌局的自动识别与输入,进一步提升系统的实用性。并且引入自动化工作流,实现牌例由图像到识别到求解的全流程自动化。


    \indent其次,当前系统的用户界面设计较为基础,未来计划引入更先进的前端框架,如React
    或Vue,以提升界面的交互性与视觉效果,增强用户体验。

    \indent此外,考虑到登录账户信息与比赛结果数据传递的安全性,未来计划引入多因素认证与加密存储等技术,提升系统的安全防护能力,确保用户数据的隐私与安全。同时,碍于开发时间的限制,针对成员数据库信息的校验功能尚未完善,未来计划引入更为严格的数据校验机制,确保用户信息的准确性与完整性。

    \indent最后,未来计划引入更多智能化功能,如基于机器学习的叫牌建议系统,通过分析大量历史牌局数据,为用户提供更为精准的叫牌策略建议,进一步提升系统的智能化水平与实用价值,使其能够在未能获取完整局面信息的情况下为学习者提供合理建议。
    另外,尽管AI在训练时可能涉及桥牌相关数据,但目前尚无成熟的AI桥牌教学辅助工具问世。AI在此领域的应用仍处于初级阶段,缺乏针对性的实用工具。未来,我们计划探索将生成式AI技术更深入地融入系统中,开发出更智能化的桥牌教学辅助工具,以满足用户日益增长的需求。

\section{分工说明}
    \indent刘广易:负责前端界面设计、Trie树提示逻辑开发、比赛结果提交及审核系统及报告撰写。
    
    \indent许昊天:负责后端flask架构搭建、IMP及Elo计算引擎开发、数据库设计及积分重算逻辑实现。
\section{生成式ai使用说明}
    \indent在前端开发上,由于前端技术尤其是css样式实现相对繁难,在项目初期利用ai构建了html文件框架,并在后续的css样式编写中,使用ai生成了部分样式代码片段,提升了开发效率。

    \indent利用Ai将一些复杂的计算逻辑如IMP转换等从Python翻译为Javascript代码,实现了在用户本地处理相关运算的能力,并避免了复杂逻辑的翻译失真。

    \indent在测试阶段,利用生成式ai工具自动化生成了多种极端定约组合的测试用例库,并将其与人工核算数据进行交叉比对,确保了系统逻辑的准确性与鲁棒性。
    
\section{参考文献}
\begin{thebibliography}{99} % “99”是预留宽度,表示文献编号最多两位数

\bibitem{zheng bridge} 
郑惠南. \emph{Bridge桥牌课程ppt}. 2025. [PDF 文件].
\bibitem{oiwiki-trie} 
OI Wiki. 字典树 (Trie) [EB/OL]. url{https://oi-wiki.org/string/trie/}, 2025.

\bibitem{dds-bridge} 
DDS-Bridge. dds: A double dummy solver for the game of bridge written in C++ [SW/OL]. url{https://github.com/dds-bridge/dds}, 2018.

\end{thebibliography}
\end{document}